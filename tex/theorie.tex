\chapter{Literature Review (work title)} \label{theorie}

% A comprehensive review of existing research relevant to your topic. This section establishes the foundation on which your research is built and identifies gaps your study aims to fill.

\section{Introduction to Antibodies}

% Basic Immunology Concepts: Overview of the immune system, with a focus on the structure and function of antibodies. Antibody Diversity: Discuss how diversity is generated in antibody populations, emphasizing the significance of heavy and light chains.

Antibodies are Y-shaped proteins that consist of two identical light chains (LCs) and two identical heavy chains (HCs) \citep{Chiu2019}.


\section{Antibody Engineering and Therapeutic Applications}

% Overview: Introduction to the field of antibody engineering and the role of antibodies in therapeutic applications. Challenges: Discuss the challenges in identifying and pairing heavy and light chains in recombinant antibodies and their implications for therapeutic development.

\section{Deep Learning Methods for Antibodies}

Deep learning is a branch of machine learning that focuses on algorithms capable of identifying complex patterns in data by transforming low-level inputs (like pixels in an image) into high-level features (such as object shapes). It utilizes artificial neural networks (ANNs) with multiple layers between the input and output, making them "deep". These networks consist of nodes, or neurons, that process inputs and pass the outputs to subsequent layers, gradually extracting more abstract features. In the context of biochemistry, deep learning can start from basic data, like amino acid sequences, and learn to recognize complex biological structures or functions \citep{Graves2020}.

\section{BERT and Transformers in Bioinformatics}

% BERT Overview: Explain the BERT model (Bidirectional Encoder Representations from Transformers) and its significance in natural language processing (NLP). Adaptation to Bioinformatics: Discuss how BERT and transformer models have been adapted for bioinformatics applications, including examples where they have been used for sequence analysis, protein function prediction, etc.



\section{Heavy-Light Chain Pair Identification}

% Current Methods: Review current methods and challenges in identifying and pairing heavy and light chains in antibodies. Potential of BERT Models: Discuss the potential advantages and challenges of using BERT models for heavy-light chain pair identification, including any preliminary findings or hypotheses.


\subsection{Gap in the Literature}

% Identify the Gap: Based on your review, identify the gap in current research that your thesis aims to address. This could involve limitations in existing methods for heavy-light chain identification, or the unexplored potential of BERT models in this specific application.



\subsection{Conclusion}

% Summarize Key Points: Briefly summarize the key points from your literature review, emphasizing the importance of your research question and the potential impact of your work.

